\documentclass[11pt,a4paper]{article}

\usepackage[a4paper,margin=1in]{geometry}
\usepackage{titlesec}
\usepackage{hyperref}
\usepackage{enumitem}
\usepackage{listings}
\usepackage{xcolor}
\usepackage{parskip}
\usepackage{graphicx}

\setlength{\parindent}{0pt}

\definecolor{lightgray}{gray}{0.95}

\lstset{
  backgroundcolor=\color{lightgray},
  basicstyle=\ttfamily\small,
  breaklines=true,
  frame=single,
  showstringspaces=false,
  columns=fullflexible
}

\titleformat{\section}{\Large\bfseries}{\thesection}{1em}{}
\titleformat{\subsection}{\large\bfseries}{\thesubsection}{1em}{}

\title{\textbf{MICS Open Network Security}\\[0.5em]
Practical Intro to Ethical Hacking}
\author{Michal Sterzel, Alberto Finardi, Tom Gave, Jan Marxen \\ \small Secan-Lab.uni.lu}
\date{Winter 2025/2026}

\begin{document}

\maketitle
\tableofcontents
\newpage

\section{Introduction}
This short guide documents the "Open Network Security" lab that is
included in this repository. It explains how to install the required
software on the host, how to start the lab VMs with Vagrant, 
and how to verify that the DNS/DHCP services are running correctly.

The lab provided here uses two VM roles: a DNS-DHCP server and an Agent
client. The participant performs detective tasks on the DNS-DHCP VM by
observing logs and DNS traffic.

\textit{Updated: October 31, 2025}

\textbf{Contributors:} Michal Sterzel, Alberto Finardi, Tom Gave,
Jan Marxen

\section{Important disclaimer}
An Ethical Hacking lab must be isolated and used responsibly. Before
running any experiment, ensure you follow these rules:

\begin{itemize}[noitemsep]
  \item Do all exercises only inside an isolated lab environment.
  \item Do not connect lab network interfaces to production networks.
  \item Do not run offensive or scanning tools against external hosts.
\end{itemize}

Failure to follow these rules can have severe legal
consequences.

\section{Prerequisites}
The lab requires the following software on the host machine:

\begin{itemize}[noitemsep]
  \item VirtualBox -- the lab uses VirtualBox as provider.
  \item Vagrant -- used to define and run the VMs.
  \item Ansible -- required for provisioning with provided host-based playbooks.
\end{itemize}

The repository contains installer helpers that attempt to install missing
prerequisites automatically:

\begin{itemize}[noitemsep]
  \item Linux/macOS: \texttt{./install-prerequisites.sh}
  \item Windows: \texttt{install-prerequisites.bat}
\end{itemize}

Run the appropriate script for your platform. The Windows script will
advise on Ansible installation options (WSL / Python) rather than forcing
one method.

\section{Lab setup}
Starting the lab is intentionally simple. From the repository root run
the launcher, which brings up and provisions the VMs in the correct
order.

\begin{itemize}[noitemsep]
  \item Linux/macOS: \texttt{./launch-lab.sh}
  \item Windows: \texttt{launch-lab.bat}
\end{itemize}

The launcher starts the DNS-DHCP VM first, then the Agent VM. Ansible
playbooks in the repository provision the services: dnsmasq for the server,
and client configuration with necessary covert communication abilities. 
The launcher includes a "clean" option to destroy and
recreate VMs (\texttt{./launch-lab.sh clean}) and a "noclean" option to
keep existing VMs.

After the launcher completes, verify the lab using the helper:

\begin{verbatim}
chmod +x test-lab.sh
./test-lab.sh
\end{verbatim}

or (windows):

\begin{verbatim}
test-lab.bat
\end{verbatim}

The helper prints the DHCP-assigned address for the Agent and performs
basic connectivity checks to the DNS server. You can also test manually
from the Agent VM after \texttt{vagrant ssh}:

\begin{verbatim}
ip addr show
ip route show
ping -c 4 192.168.10.1    # ping DNS-DHCP server
nslookup example.com 192.168.10.1
\end{verbatim}

To shutdown the lab, run the following commands from the repository root:
\begin{verbatim}
chmod +x shutdown-lab.sh
./shutdown-lab.sh
\end{verbatim}

or (windows):

\begin{verbatim}
./shutdown-lab.bat
\end{verbatim}

In summary, the workflow is:
\begin{enumerate}[noitemsep]
  \item Install prerequisites \texttt{./install-prerequisites.sh} (if needed).
  \item Launch the lab: \texttt{./launch-lab.sh}
  \item Verify the lab: \texttt{./test-lab.sh}
  \item Perform exercises inside the lab.
  \item Shutdown the lab: \texttt{./shutdown-lab.sh}
\end{enumerate}

or (windows):

\begin{enumerate}[noitemsep]
  \item Install prerequisites \texttt{install-prerequisites.bat} (if needed).
  \item Launch the lab: \texttt{launch-lab.bat}
  \item Verify the lab: \texttt{test-lab.bat}
  \item Perform exercises inside the lab.
  \item Shutdown the lab: \texttt{shutdown-lab.bat}
\end{enumerate}

\section{VM roles and lab model}
The lab uses two VM roles:

\begin{itemize}[noitemsep]
  \item \textbf{DNS-DHCP VM} -- runs \texttt{dnsmasq} on the static
    address \texttt{192.168.10.1}. It serves DHCP leases (default range
    \texttt{192.168.10.3-192.168.10.100}) and logs incoming DNS queries.
  \item \textbf{Agent VM} -- a DHCP client that simulates covert
    communications by sending encoded DNS queries to the DNS server.
\end{itemize}

In the exercises the participant acts as the detective by inspecting DNS
requests and logs on the DNS-DHCP VM to identify and decode covert
messages sent by the Agent.

\section{Notes on network isolation and safety}
All lab VMs attach to a VirtualBox internal network called
\texttt{labnet}. This keeps lab traffic inside your host machine. Do
not add bridged adapters to lab interfaces; keep them internal or
host-only.

If you want to ensure guests have no Internet access, add firewall
rules inside the guests. The Ansible playbooks include commented
examples for adding restrictive iptables/nftables rules.

\section{Troubleshooting}
Common checks and commands:

\begin{itemize}[noitemsep]
  \item On the host: run \texttt{vagrant status} in each VM folder to
    verify VM states.
  \item On the DNS-DHCP VM (via \texttt{vagrant ssh}):
    \texttt{sudo systemctl status dnsmasq} and
    \texttt{sudo journalctl -u dnsmasq -n 200}.
  \item On the Agent VM: check IP and resolution with \texttt{ip addr},
    \texttt{ip route}, \texttt{cat /etc/resolv.conf}, and
    \texttt{ping 192.168.10.1}.
\end{itemize}

Security reminder: perform all exercises only inside this isolated lab
and do not direct lab traffic to other networks.

\section{Further reading}
\begin{itemize}
  \item Project README and playbooks in this repository (start here).
  \item dnsmasq man pages and documentation: \url{http://www.thekelleys.org.uk/dnsmasq/doc.html}
  \item Ansible documentation: \url{https://docs.ansible.com/}
\end{itemize}

\end{document}
